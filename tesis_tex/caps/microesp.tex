\singlespacing
\chapter{Microespectrómetro}
\label{chap:microsp}
\spacing{1.5}

\hspace{0.5cm}En el Capítulo \ref{chap:zeiss} se realizó una descripción cuantitativa de los defectos: se determinó su área, su diámetro equivalente, la cantidad de defectos presentes en cada banda, etc. Dicho análisis permitió entender las especificaciones técnicas de \textit{scratch \& dig} y de la ISO 10110, lo que resulta fundamental para establecer las bases de futuros acuerdos con los fabricantes de los componentes ópticos. Al mismo tiempo, los resultados de la población de defectos en general [\ref{sec:defpob}] detectados con el algoritmo permitieron establecer los criterios de diseño óptico del microespectrómetro que se explica en el presente capítulo.

Como se explicó en el Capítulo \ref{chap:introd}, un microespectrómetro es un instrumento de medición híbrido que integra la capacidad de magnificación y de resolución ópticas de un microscopio con la capacidad de inferir las propiedades ópticas de un material de un espectrómetro. En este capítulo se describe la construcción y montaje de un microespectrómetro que permitió realizar una caracterización de las propiedades ópticas del filtro y de sus defectos a través de los espectros de transmisión [\ref{sec:montcontmsp}]. En la Sección \ref{sec:prot0} muestran las características del primer prototipo desarrollado con equipamiento disponible en el laboratorio. Dicho prototipo en conjunto con los resultados y análisis del Capítulo \ref{chap:zeiss} permitieron establecer los criterios de elección de la fuente de luz y del espectrómetro [\ref{sec:fteluzyesp}], determinar la longitud del recorrido y la precisión mínima necesaria de la platina que se desarrolló [\ref{sec:platina}] y la resolución óptica necesaria del microscopio desarrollado para caracterizar los defectos de diámetro mayores a 20$\mu m$ de diámetro [\ref{sec:disop}]. Luego se explica el proceso de montaje y alineación preliminar del microespectrómetro [\ref{sec:montalin}] así como su puesta en foco y la determinación de la resolución espacial [\ref{sec:focoresol}]. Posteriormente se explica la integración de una cámara web al microespectrómetro lo que permitió la adquisición simultánea de imágenes digitales y de espectros de transmisión y cuya área de adquisición fue elegida con un \textit{joystick} y visualizada en vivo a través de una interfaz gráfica [\ref{sec:camwebgui}].

Asimismo, se muestran los resultados de los espectros de transmisión de cada banda del filtro y su comparación con la hoja de datos reportada por el fabricante [\ref{sec:espectransm}]. Finalmente, se muestran los resultados de la caracterización espectral de los defectos denominados manchas ó defectos de transmisión [\ref{sec:defctma}] y de los agujeros ó huecos [\ref{sec:defctag}].

%%%%%%%%%%%%%%%%%%%%%%%%%%%%%%%%%%%%%%%%%%%%%%%%%%%%%%%%%%%%%%%%%%%%%%%%%%%%%%%%%%%%%%%%%%%%%%%%%%%%%%%%%%%%%%%%%%%%%%%%%%%%%%%%%%%%%%%%%%%%%%%%%%%%%%%%%%%%%%%%%%%%%%%%%%%%%%%%%%%%%%%%%%%%%%%%%%%%%%%%%%%%%%%%%%%%%%%%%%%%

\singlespacing
\section{Prototipo 0 \href{https://github.com/jrr1984/thorlabs_step_motors_ZST213B/tree/master/barrido/std}{\faGithub}}
\label{sec:prot0}
\spacing{1.5}

\hspace{0.5cm}El desarrollo del prototipo 0 que se muestra en esta sección permitió por un lado establecer los criterios de elección de la fuente de luz y del recorrido y la precisión necesarias de la platina desarrollada para poder adquirir el espectro e imágenes del filtro completo. Como buena práctica de prototipado de instrumentos de medición se utilizaron componentes y equipamiento disponibles en el laboratorio, es decir que no se incurrió en gastos adicionales de dinero a excepción del costo del material de las impresiones 3D del soporte del filtro. Por otro lado, a partir del desarrollo del software automatizado de adquisición y de visualización de los resultados de este prototipo se establecieron las características deseadas y esperadas del prototipo final desarrollado. De esta manera, el prototipo permitió establecer la factibilidad del desarrollo del equipo.

Como objetivo general se propuso desarrollar un sistema integral de caracterización de filtros ópticos de interferencia utilizados en cámaras hiper y multiespectrales. Inicialmente se propusieron tres objetivos específicos:
\begin{enumerate}
\item Desarrollar un sistema automatizado de adquisición del espectro de transmisión de cada una de las bandas del filtro.$\xrightarrow{}$ \href{https://github.com/jrr1984/thorlabs\_step\_motors\_ZST213B/tree/master/barrido/std}{\faGithub}

\item Determinar un mapa hiperespectral ($\textit{x}$,$\textit{y}$,$\lambda$) del filtro.$\xrightarrow{}$ \href{https://github.com/jrr1984/thorlabs_step_motors_ZST213B/blob/master/spectral_gui/main.py}{\faGithub}

\item Integrar un sistema de detección y caracterización de los defectos del filtro.
\end{enumerate}

A continuación se describen de forma resumida los primeros dos objetivos que fueron abarcados por el prototipo 0 descripto en esta sección. Respecto del primer objetivo, se montó el arreglo experimental que se muestra en las Figuras \ref{fig:setup0}, \ref{fig:setup01} y \ref{fig:setup02}. 

\begin{figure}[H]
	\centering
	\includegraphics[width=1.0\textwidth]{Figs/microespectrometro/setupbarridooriginal.jpg}
	\caption{Arreglo experimental del prototipo 0.}
	\label{fig:setup0}
\end{figure}

\begin{figure}[H]
	\begin{floatrow}
		\ffigbox{\includegraphics[scale=0.073]{Figs/microespectrometro/montajesetup0.jpg}}{\caption{Vista lateral del montaje del filtro sobre el soporte que se encuentra atornillado con unos tornillos M6 a la plataforma motorizada de Thorlabs.}\label{fig:setup01}}
		\ffigbox{\includegraphics[scale=0.073]{Figs/microespectrometro/5.jpg}}{\caption{El filtro se mueve en los ejes $\textit{x}$ e $\textit{y}$. La fuente de luz y la fibra del espectrómetro se encuentran inmóviles. }\label{fig:setup02}}
	\end{floatrow}
\end{figure}

Con una fuente de luz halógena, modelo \href{https://dolan-jenner.com/products/fiber-lite-190}{\textit{Fiber-lite 190 Illuminator}}, se incidió perpendicularmente sobre el filtro y su transmisión fue detectada por la fibra óptica de un espectrómetro modelo \href{https://www.thorlabs.com/thorproduct.cfm?partnumber=CCS200/M#ad-image-0}{CCS 200/M} de la empresa Thorlabs (\textit{Driver} de \textit{python}: \href{https://github.com/jrr1984/thorlabs_step_motors_ZST213B/blob/master/syst/CCS200.py}{\faGithub}). De la misma manera en que se realizó el \textit{Tile Scan} del filtro completo con el microscopio Zeiss explicado en la sección \ref{subs:tilsc} (Ver Figura \ref{fig:tilescan}), se desplazó el filtro a lo largo del eje x, barriendo en `filas' y realizando el desplazamiento vertical en los extremos del máximo recorrido de los tornillos accionados por los motores paso a paso que fue de 13 mm. Dicho desplazamiento fue realizado con una plataforma de tres grados de libertad, modelo \href{https://www.thorlabs.com/thorproduct.cfm?partnumber=MT3/M}{MT3/M} de la empresa Thorlabs, cuyos tornillos micrométricos fueron intercambiados por unos motores paso a paso modelo \href{https://www.thorlabs.com/thorproduct.cfm?partnumber=ZST213B}{ZST213B} cuyos controladores fueron también de la empresa Thorlabs, modelo \href{https://www.thorlabs.com/thorproduct.cfm?partnumber=KST101}{KST101} (\textit{Driver} de \textit{python}: \href{https://github.com/jrr1984/thorlabs_step_motors_ZST213B/blob/master/barrido/std/thor_stepm.py}{\faGithub}). Como las dimensiones de la región comprendida por las cinco bandas del filtro es de 27 mm x 25 mm, con esta plataforma no se pudo realizar una adquisición del filtro completa en una sola configuración como la propuesta. El \textit{software} automatizado de adquisición del espectro de transmisión del filtro desarrollado para este prototipo [\href{https://github.com/jrr1984/thorlabs\_step\_motors\_ZST213B/tree/master/barrido/std}{\faGithub}] fue expandido en el prototipo final y se lo explica en la Sección \ref{sec:softadq}.


No se utilizó ningún arreglo óptico ni para enfocar la fuente de luz en el filtro ni para enfocar su transmisión divergente sobre la fibra óptica del espectrómetro. No se caracterizó la resolución óptica con la que se realizaron las mediciones con el espectrómetro en este prototipo ni el tamaño del objeto medido sobre la superficie del filtro. La resolución óptica y magnificación necesarias para medir los defectos del filtro fueron establecidas a partir de los resultados del Capítulo \ref{chap:zeiss} y fueron consideradas en el diseño óptico del microespectrómetro desarrollado que se explica en la Sección \ref{sec:disop}.


Respecto del segundo objetivo específico propuesto relacionado con la determinación de un mapa hiperespectral ($\textit{x}$,$\textit{y}$,$\lambda$) del filtro, se adquirió el espectro de transmisión de una región del filtro con dimensiones iguales 13 mm en el eje $\textit{x}$ y de 24.6 mm a lo largo del eje $\textit{y}$ (las cinco bandas junto al cromo que las separa tienen una altura de 25 mm, ver Figura \ref{fig:dimsfiltr}). El área del filtro adquirida fue el resultado de unir las mediciones de dos barridos cuyas dimensiones fueron para cada uno, de 13 mm a lo largo del eje $\textit{x}$ y de 12.2 mm a lo largo del eje $\textit{y}$. Los ejes fueron definidos de acuerdo al sistema de coordenadas de la Figura \ref{fig:setup02} y el paso del desplazamiento de los motores de cada eje fue de 50 $\mu m$. La adquisición fue realizada en dos etapas debido a la limitación del recorrido de los tornillos desplazados por los motores paso a paso como se explicó anteriormente. De esta manera se adquirió en primer lugar la región superior del filtro que contiene a las bandas azul, verde y parte de la pancromática con una cierta altura de la fuente de luz y de la fibra del espectrómetro. La fuente y la fibra se encontraban montadas sobre unos posicionadores micrométricos con los cuales se varió su altura respecto del filtro para poder medir la región inferior del filtro que contenía la región faltante de la banda pancromática, la banda roja y la banda del NIR. La fuente de luz y la fibra del espectrómetro fueron posicionadas lo más cerca posible del filtro, sin intervenir su libre desplazamiento para realizar el barrido, con el fin de minimizar el tiempo de integración de cada medición del espectro de transmisión que fue de 1 ms.

Con el objetivo de visualizar una imagen completa del filtro como la que se obtuvo con el microscopio Zeiss (Ver Figura \ref{fig:supfiltrocondensador}) pero que contenga además la información espectral de cada medición se desarrolló una interfaz gráfica interactiva que se muestra en la Figura \ref{fig:GUI00}.

\begin{figure}[H]
	\centering
	\includegraphics[width=1.0\textwidth]{Figs/microespectrometro/guirgb.png}
	\caption{Interfaz gráfica con el \textit{imshow} en RGB.}
	\label{fig:GUI00}
\end{figure}


La interfaz gráfica fue realizada [\href{https://github.com/jrr1984/thorlabs\_step\_motors\_ZST213B/blob/master/spectral\_gui/main.py}{\faGithub}] con la librería	 \href{https://wiki.python.org/moin/TkInter}{\textit{Tkinter}}. El barrido completo que se muestra en la imagen del mapa de colores de la Figura \ref{fig:GUI00} estuvo compuesto por 127920 mediciones de espectros de transmisión, que tomaron aproximadamente 18 horas de medición en total (Control remoto de la computadora con \href{https://anydesk.com/es}{AnyDesk} y control del experimento vía \href{https://pypi.org/project/cutelog/}{cutelog}, ver Sección \ref{sec:softadq}). Con la librería \href{https://pypi.org/project/colorpy/}{ColorPy} se obtuvo una tupla RGB a partir del espectro medido y cada tupla fue asignada a una cierta posición del filtro. El conjunto total de todas las posiciones del filtro medidas estuvo formado por una matriz de 492 filas y 260 columnas. Dicha matriz fue mostrada como una imagen RGB con el método \textit{imshow} de la librería \textit{Matplotlib}. El gráfico debajo del \textit{imshow} muestra el espectro medido (gráfico de Intensidad en función de la longitud de onda) para el punto de la imagen sobre el cual se posicione el \textit{mouse} de forma actualizada y no muestra nada si se posiciona el mouse fuera de la imagen. 

En lugar de la imagen RGB de las mediciones también se puede mostrar el $\chi^{2}$ del espectro de transmisión de cada banda lo que permitiría ver la homogeneidad del espectro de transmisión de cada banda, que fue definido para la i-ésima medición de cada banda de la siguiente manera:
\begin{equation}
\chi^{2}_{banda}(i) = \sum \frac{(medici\acute{o}n_{i} - espectro\_medio_{banda})^{2}}{medici\acute{o}n^{2}_{i}}
\end{equation}
donde $espectro\_medio_{banda}$ es el resultado de tomar el valor medio de todas las mediciones de una banda. En la Figura \ref{fig:GUI01} se muestra una región del filtro que contiene al cromo (en color amarillo) que separa las bandas pancromática de la banda verde.
\begin{figure}[H]
	\centering
	\includegraphics[width=1.0\textwidth]{Figs/microespectrometro/chidisp.png}
	\caption{Interfaz gráfica con el \textit{imshow} del $\chi^{2}$ de cada banda. \textit{Zoom} sobre el cromo que separa las bandas pancromática de la banda verde.}
	\label{fig:GUI01}
\end{figure}

La interfaz gráfica permite además graficar el espectro de un cierto píxel de la imagen generada a partir de la selección con el mouse, guardar imágenes de la región de interés, mostrar el espectro de la fuente de luz utilizada, etc, todas opciones que se consideraron útiles para la caracterización de las propiedades ópticas del filtro y deseadas para el prototipo final del equipo.

El prototipo 0 permitió establecer las características deseadas del equipo final y evaluar su factibilidad sin incurrir en gastos importantes de prototipado. Ahora bien, debido a la falta de resolución óptica no se obtuvo ningún resultado concluyente ya que este prototipo resultó una prueba de concepto. A este prototipo se le propusieron las siguientes mejoras que fueron incluidas en el montaje y construcción del microespectrómetro [\ref{sec:montcontmsp}]:

\begin{enumerate}
\justifying
\item \texttt{Fuente de luz [\ref{sec:fteluzyesp}]}: Se modificó la fuente de luz de \textit{Fiber-Lite} que consiste de un \textit{fiber bundle} (arreglo de fibras ópticas) de un diámetro de 4.8 mm por una fuente de luz acoplada con una fibra óptica multimodo cuya apertura numérica fue de 0.22 y cuyo diámetro del \textit{core} fue de 200 $\mu m$. Como no se contó con un objetivo adicional para enfocar la fuente de luz sobre el filtro y mediante un arreglo óptico elegir el tamaño del \textit{spot} incidente sobre el filtro , se decidió optar por la fuente de luz cuya salida divergente tuviera el menor ángulo del cono de luz de salida. Este criterio de diseño tenía como objetivo disminuir la región iluminada del filtro por la fuente de luz, lo que reduciría ciertas reflexiones espurias en los componentes ópticos del microespectrómetro provenientes de la luz de regiones no alcanzadas por el área de adquisición del microespectrómetro. Este efecto que aumenta proporcionalmente a la relación entre el área iluminada y el área adquirida del filtro se denomina efecto de Schwarzchild-Villiger \cite{Naora279} debería ser considerado en las futuras mejoras del equipo aquí propuesto ya que distorsiona el espectro de la región original que se quiere medir. Además la nueva fuente de luz utilizada tenía la misma fibra óptica que el espectrómetro, hecho que permitió realizar fácilmente la identificación de la región medida con el espectrómetro respecto de la imagen adquirida con la cámara web (Ver Sección \ref{sec:camwebgui}).
\item \texttt{Platina [\ref{sec:platina}]}: Como la plataforma motorizada de Thorlabs utilizada en el prototipo 0 tenía un límite de recorrido de 13 mm x 13 mm, no se podía adquirir el espectro de transmisión del filtro completo cuya región que contiene a las cinco bandas tiene unas dimensiones de 27 mm x 25 mm. En consecuencia se desarrolló una platina motorizada con el suficiente recorrido para poder realizar el barrido completo del filtro y también se consideró el paso y precisión mecánicas mínimas como para poder adquirir áreas de defectos de diámetro mayor a 20$\mu m$ con la mayor cantidad de puntos posible.
\item \texttt{Microespectrómetro [\ref{sec:disop}-\ref{sec:softadq}]}: Se montó e incorporó al prototipo un microespectrómetro con una resolución óptica lateral diseñado para caracterizar defectos de diámetro mayor a 20 $\mu m$.
\item \texttt{Integración de una cámara web [\ref{sec:camwebgui}]}: Se incorporó una cámara \textit{web} y un  \textit{joystick} al equipo para poder seleccionar la región del filtro a medir y se desarrolló una interfaz gráfica para poder visualizar en simultáneo la imagen digital de dicha región y el espectro de transmisión. 
\end{enumerate}

%%%%%%%%%%%%%%%%%%%%%%%%%%%%%%%%%%%%%%%%%%%%%%%%%%%%%%%%%%%%%%%%%%%%%%%%%%%%%%%%%%%%%%%%%%%%%%%%%%%%%%%%%%%%%%%%%%%%%%%%%%%%%%%%%%%%%%%%%%%%%%%%%%%%%%%%%%%%%%%%%%%%%%%%%%%%%%%%%%%%%%%%%%%%%%%%%%%%%%%%%%%%%%%%%%%%%%%%%%%%

\singlespacing
\section{Montaje y construcción del microespectrómetro}
\label{sec:montcontmsp}
\spacing{1.5}

\hspace{0.5cm}En esta sección se describen los criterios de diseño y todas las consideraciones técnicas del microespectrómetro, de la platina motorizada desarrollada que fue controlada con un \textit{joystick} y de la cámara \textit{web} integrada. El equipo final desarrollado puede ser fácilmente adaptable a requerimientos ópticos y mecánicos específicos distintos a los presentados en esta tesis.
%%%%%%%%%%%%%%%%%%%%%%%%%%%%%%%%%%%%%%%%%%%%%%%%%%%%%%%%%%%%%%%%%%%%%%%%%%%%%%%%%%%%%%%%%%%%%%%%%%%%%%%%%%%%%%%%%%%%%%%%%%%%%%%%%%%%%%%%%%%%%%%%%%%%%%%%%%%%%%%%%%%%%%%%%%%%%%%%%%%%%%%%%%%%%%%%%%%%%%%%%%%%%%%%%%%%%%%%%%%%

\singlespacing
\subsection{Fuente de luz y espectrómetro}
\label{sec:fteluzyesp}
\spacing{1.5}

\hspace{0.5cm}El criterio de elección de la fuente de luz depende fundamentalmente del rango de longitudes de onda que se quiera medir, que para el caso del filtro aquí analizado dicho rango se encuentra entre los 450 nm y los 900 nm.
Se utilizó una fuente de luz halógena y de tungsteno modelo \href{https://www.thorlabs.com/newgrouppage9.cfm?objectgroup_id=7269&pn=SLS201L/M}{SLS201L} del fabricante Thorlabs
%%%%%%%%%%%%%%%%%%%%%%%%%%%%%%%%%%%%%%%%%%%%%%%%%%%%%%%%%%%%%%%%%%%%%%%%%%%%%%%%%%%%%%%%%%%%%%%%%%%%%%%%%%%%%%%%%%%%%%%%%%%%%%%%%%%%%%%%%%%%%%%%%%%%%%%%%%%%%%%%%%%%%%%%%%%%%%%%%%%%%%%%%%%%%%%%%%%%%%%%%%%%%%%%%%%%%%%%%%%%

\singlespacing
\subsection{Platina}
\label{sec:platina}
\spacing{1.5}

%%%%%%%%%%%%%%%%%%%%%%%%%%%%%%%%%%%%%%%%%%%%%%%%%%%%%%%%%%%%%%%%%%%%%%%%%%%%%%%%%%%%%%%%%%%%%%%%%%%%%%%%%%%%%%%%%%%%%%%%%%%%%%%%%%%%%%%%%%%%%%%%%%%%%%%%%%%%%%%%%%%%%%%%%%%%%%%%%%%%%%%%%%%%%%%%%%%%%%%%%%%%%%%%%%%%%%%%%%%%

\singlespacing
\subsection{Diseño óptico del microespectrómetro}
\label{sec:disop}
\spacing{1.5}

%%%%%%%%%%%%%%%%%%%%%%%%%%%%%%%%%%%%%%%%%%%%%%%%%%%%%%%%%%%%%%%%%%%%%%%%%%%%%%%%%%%%%%%%%%%%%%%%%%%%%%%%%%%%%%%%%%%%%%%%%%%%%%%%%%%%%%%%%%%%%%%%%%%%%%%%%%%%%%%%%%%%%%%%%%%%%%%%%%%%%%%%%%%%%%%%%%%%%%%%%%%%%%%%%%%%%%%%%%%%

\singlespacing
\subsection{Montaje y alineación preliminar del microespectrómetro}
\label{sec:montalin}
\spacing{1.5}

%%%%%%%%%%%%%%%%%%%%%%%%%%%%%%%%%%%%%%%%%%%%%%%%%%%%%%%%%%%%%%%%%%%%%%%%%%%%%%%%%%%%%%%%%%%%%%%%%%%%%%%%%%%%%%%%%%%%%%%%%%%%%%%%%%%%%%%%%%%%%%%%%%%%%%%%%%%%%%%%%%%%%%%%%%%%%%%%%%%%%%%%%%%%%%%%%%%%%%%%%%%%%%%%%%%%%%%%%%%%

\singlespacing
\subsection{Foco y resolución espacial del microespectrómetro}
\label{sec:focoresol}
\spacing{1.5}

\hspace{0.5cm}Para poner en foco el microespectrómetro sobre la cara externa del filtro más cerca al objetivo, se buscó el mínimo de la resolución espacial.

La resolución espacial se obtiene a partir del ajuste de las mediciones de una transición banda-cromo.

Para no alargar el tiempo de duración de las mediciones se mapeó el espectrómetro con la cámara. De lo contrario el único feedback que se tiene para saber si se está en una banda o en el cromo es la medición del espectro.

En consecuencia se conectó la fibra óptica montada sobre el cage destinado a medir con el espectrómetro, a la fuente de luz y por reflexión se observó en la adquisición en vivo de la cámara en qué posición de la imagen se observaba el haz de luz reflejado. Se centró dicho haz al centro de la cámara y de esa forma se determinó que el centro de la cámara está asociado con la medición efectiva del microespectrómetro. Se hace notar que la cámara no se encuentra en foco todavía, solo fue puesta aproximadamente a la misma distancia focal que la lente de tubo.

El setup para realizar este mapeo es el siguiente:
\begin{figure}[H]
	\centering
	\includegraphics[scale=0.1]{Figs/microespectrometro/mapespeccam.jpg}
	\caption{Setup para mapear el espectrómetro con la cámara.}
	\label{fig:bgcel}
\end{figure}


Con los tornillos de la tapa de arriba del beamsplitter se puede ajustar en altura el beamsplitter para poder observar en el centro de la cámara la medición del espectrómetro.
\begin{figure}[H]
	\centering
	\includegraphics[scale=0.5]{Figs/microespectrometro/b4c.png}
	\caption{Tapa de arriba del beamsplitter.}
	\label{fig:bgcel}
\end{figure}


\begin{figure}[H]
	\centering
	\includegraphics[scale=0.5]{Figs/microespectrometro/mapspectrometrocamera.png}
	\caption{Visualización en la cámara de la reflexión del filtro de la iluminación.}
	\label{fig:bgcel}
\end{figure}


No se tocó ni la cámara ni ninguna parte del setup a partir de ese momento para no perder este mapeo, a pesar de que la cámara no se encuentre perfectamente en foco (no hace falta probablemente poner una imagen de la cámara mostrando que no está en foco..), es decir que la imagen no se vea del todo nítida.
Luego se puso en foco el microespectrómetro.



\begin{figure}[H]
	\centering
	\includegraphics[scale=0.5]{Figs/microespectrometro/medtransicion.png}
	\caption{Visualización en la cámara de la reflexión del filtro de la iluminación.}
	\label{fig:bgcel}
\end{figure}


durante el experimento se tiene el feedback de cutelog:

\begin{figure}[H]
	\centering
	\includegraphics[scale=0.5]{Figs/microespectrometro/cutelog.png}
	\caption{Visualización en la cámara de la reflexión del filtro de la iluminación.}
	\label{fig:bgcel}
\end{figure}


\begin{figure}[H]
	\centering
	\includegraphics[scale=0.5]{Figs/microespectrometro/fincutelog.png}
	\caption{Visualización en la cámara de la reflexión del filtro de la iluminación.}
	\label{fig:bgcel}
\end{figure}


Las mediciones son ajustadas en matlab con una función error:
\begin{equation}
	(a/2)*erfc(sqrt(2)*(x-b)/c)
\end{equation}

\begin{figure}[H]
	\centering
	\includegraphics[scale=0.3]{Figs/microespectrometro/fit0.png}
	\caption{Visualización en la cámara de la reflexión del filtro de la iluminación.}
	\label{fig:bgcel}
\end{figure}

Resultados del ajuste:

General model:\par
$f(x) = (a/2)*erfc(sqrt(2)*(x-b)/c)$ \par
Coefficients (with 95$\%$ confidence bounds): \par
$a =       42.43  (42.2, 42.66)$\par
$b =       108.2  (107.8, 108.7)$\par
$c =       50.46  (49.25, 51.68)$\par

Goodness of fit:\par
SSE: 151.1\par
R-square: 0.9976\par
Adjusted R-square: 0.9976\par
RMSE: 0.8758\par

Luego moviendo la perilla del SM1Z para cambiar la distancia entre el objetivo y el filtro se repite la medición.


Comentar bien la siguiente foto, poner en la imagen que distancia se está variando, ettc
\begin{figure}[H]
	\centering
	\includegraphics[scale=0.4]{Figs/microespectrometro/refinacionparam.png}
	\caption{Visualización en la cámara de la reflexión del filtro de la iluminación.}
	\label{fig:bgcel}
\end{figure}


Para hacer el ajuste se refinan los parámetros del modelo:

\begin{figure}[H]
	\centering
	\includegraphics[scale=0.1]{Figs/microespectrometro/sm1zcambio.jpg}
	\caption{Visualización en la cámara de la reflexión del filtro de la iluminación.}
	\label{fig:bgcel}
\end{figure}


La idea es poner en foco el espectrómetro en alguna región del filtro, enfocar luego la cámara y después al mover el filtro a alguna otra región, tan solo hay que poner en foco el 'sistema' mirando la cámara. Al mismo tiempo si se quiere se puede volver a repetir el procedimiento de buscar el mínimo.


Gráfico de poner en foco el microespectrómetro: (19 de marzo)

mediciones guardadas en: data mediciones, simultaneidad, foco.

vamos recorriendo horario en pasos de 50 micrones en el SM1Z.
mediciones que consisten en un barrido de 80 micrones de largo, con pasos de 1 micron.. esto en la stage


RESULTADOS:

Z                  RESOLUCIÓN

0                  12.46 dudoso?

-50               13.5

-100             13.12

-150              12.87

-200              11.29

%%%%%%%%%%%%%%%%%%%%%%%%%%%%%%%%%%%%%%%%%%%%%%%%%%%%%%%%%%%%%%%%%%%%%%%%%%%%%%%%%%%%%%%%%%%%%%%%%%%%%%%%%%%%%%%%%%%%%%%%%%%%%%%%%%%%%%%%%%%%%%%%%%%%%%%%%%%%%%%%%%%%%%%%%%%%%%%%%%%%%%%%%%%%%%%%%%%%%%%%%%%%%%%%%%%%%%%%%%%%

\singlespacing
\subsection{\textit{Software} automatizado de adquisición}
\label{sec:softadq}
\spacing{1.5}

%%%%%%%%%%%%%%%%%%%%%%%%%%%%%%%%%%%%%%%%%%%%%%%%%%%%%%%%%%%%%%%%%%%%%%%%%%%%%%%%%%%%%%%%%%%%%%%%%%%%%%%%%%%%%%%%%%%%%%%%%%%%%%%%%%%%%%%%%%%%%%%%%%%%%%%%%%%%%%%%%%%%%%%%%%%%%%%%%%%%%%%%%%%%%%%%%%%%%%%%%%%%%%%%%%%%%%%%%%%%

\singlespacing
\subsection{Integración de una cámara web: adquisición simultánea de imágenes y de espectros de transmisión}
\label{sec:camwebgui}
\spacing{1.5}

%%%%%%%%%%%%%%%%%%%%%%%%%%%%%%%%%%%%%%%%%%%%%%%%%%%%%%%%%%%%%%%%%%%%%%%%%%%%%%%%%%%%%%%%%%%%%%%%%%%%%%%%%%%%%%%%%%%%%%%%%%%%%%%%%%%%%%%%%%%%%%%%%%%%%%%%%%%%%%%%%%%%%%%%%%%%%%%%%%%%%%%%%%%%%%%%%%%%%%%%%%%%%%%%%%%%%%%%%%%%

\singlespacing
\section{Resultados}
\label{sec:resgrales}
\spacing{1.5}

%%%%%%%%%%%%%%%%%%%%%%%%%%%%%%%%%%%%%%%%%%%%%%%%%%%%%%%%%%%%%%%%%%%%%%%%%%%%%%%%%%%%%%%%%%%%%%%%%%%%%%%%%%%%%%%%%%%%%%%%%%%%%%%%%%%%%%%%%%%%%%%%%%%%%%%%%%%%%%%%%%%%%%%%%%%%%%%%%%%%%%%%%%%%%%%%%%%%%%%%%%%%%%%%%%%%%%%%%%%%

\singlespacing
\subsection{Espectro de transmisión de cada banda del filtro}
\label{sec:espectransm}
\spacing{1.5}

%%%%%%%%%%%%%%%%%%%%%%%%%%%%%%%%%%%%%%%%%%%%%%%%%%%%%%%%%%%%%%%%%%%%%%%%%%%%%%%%%%%%%%%%%%%%%%%%%%%%%%%%%%%%%%%%%%%%%%%%%%%%%%%%%%%%%%%%%%%%%%%%%%%%%%%%%%%%%%%%%%%%%%%%%%%%%%%%%%%%%%%%%%%%%%%%%%%%%%%%%%%%%%%%%%%%%%%%%%%%

\singlespacing
\subsection{Caracterización espectral de las manchas ó defectos de transmisión}
\label{sec:defctma}
\spacing{1.5}

%%%%%%%%%%%%%%%%%%%%%%%%%%%%%%%%%%%%%%%%%%%%%%%%%%%%%%%%%%%%%%%%%%%%%%%%%%%%%%%%%%%%%%%%%%%%%%%%%%%%%%%%%%%%%%%%%%%%%%%%%%%%%%%%%%%%%%%%%%%%%%%%%%%%%%%%%%%%%%%%%%%%%%%%%%%%%%%%%%%%%%%%%%%%%%%%%%%%%%%%%%%%%%%%%%%%%%%%%%%%

\singlespacing
\subsection{Caracterización espectral de los agujeros ó huecos}
\label{sec:defctag}
\spacing{1.5}
