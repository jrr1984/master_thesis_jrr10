\singlespacing
\chapter{Conclusiones y trabajo a futuro}
\label{chap:concls}
\spacing{1.5}

\hspace{0.5cm}La tesis aquí presentada contribuyó al desarrollo de un prototipo de caracterización de filtros multi e hiperespectrales de cámaras satelitales a partir de la técnica de microespectroscopía, del cual no se tenían antecedentes de tesis ó trabajos anteriores de estas características en la Argentina. Parte de los desarrollos de procedimientos de caracterización de los filtros serán implementados en la empresa Argentina Satellogic, con quién se realizó la vinculación de este trabajo.

En el capítulo 2 se desarrolló un método de inspección de la calidad óptica de los filtros completamente automatizado, basado en la adquisición de imágenes de microscopía y en su posterior procesamiento y aplicación de un algoritmo de detección de los defectos. Se adquirieron imágenes de microscopía de ambas superficies exteriores del filtro lo que permitió obtener un mapa de los defectos con su ubicación precisa en cada superficie del filtro que pudo ser utilizado luego con el microespectrómetro desarrollado para poder localizar fácilmente los defectos a ser caracterizados espectralmente. Además, la medición de la longitud del cromo que separa cada par de bandas permitió realizar la calibración de la platina, utilizando dicha longitud como referencia respecto del valor medido con el microespectrómetro. Así también se hace notar que la adquisición de una imagen completa del filtro (incluida la banda del NIR) sería importante para poder identificar los futuros defectos de las imágenes adquiridas por un satélite en órbita.

Ahora bien, el algoritmo de detección de los defectos fue aplicado sobre las imágenes individuales de cada banda adquirida, bajo ciertas condiciones específicas de intensidad de la fuente de luz y del tiempo de integración de la cámara que permitieron mejorar la relación señal-ruido de las imágenes. La adquisición de cada banda fue realizada teniendo la precaución de no incluir parte del cromo que separa las bandas en el campo de visión de la cámara. Esto resultó de suma importancia pues de lo contrario el algoritmo detectaba al cromo como un centenar de defectos (falsos positivos), lo que arruinaría cualquier tipo de análisis cuantitativo.

 Para poder aplicar el algoritmo, se realizó una corrección de la iluminación no uniforme del microscopio a partir de la generación una imagen de fondo y se maximizó el contraste de la imagen a partir de la expansión del histograma de cada imagen. Esta corrección resultó fundamental para evitar la detección de múltiples falsos positivos en las regiones de cada imagen donde la iluminación no era uniforme. De esta manera, se aplicó el algoritmo de detección de los defectos por medio de técnicas de \textit{thresholding}, aplicando el umbral de Yen.
 
Se aplicaron las especificaciones técnicas de la calida óptica de la superficie del filtro de \textit{scratch \& dig} y de la norma ISO 10110, a partir de los resultados del algoritmo de detección de los defectos. Este método de inspección automatizado que consistió del desarrollo del \textit{software} de procesamiento de las imágenes de microscopía y del algoritmo de detección de los defectos, se encuentra en vías de ser implementado en la empresa Argentina Satellogic, ya que reduce los tiempos de duración de la inspección de la calidad óptica de cada filtro, reduce la ambiguedad de la inspección visual realizada por el inspector de turno y puede ser fácilmente integrado al proceso industrial de la fabricación de satélites. 

Al mismo tiempo, los resultados del análisis cuantitativo de la población de defectos permitieron establecer los criterios de diseño ópticos del microespectrómetro, de forma tal de que la resolución espacial del equipo desarrollado tuviera la suficiente resolución como para detectar defectos de diámetro mayores a los 20 $\mu m$, que constituyen la minoría de los defectos y los que mayores problemas pueden ocasionar en las imágenes adquiridas con cámaras satelitales.

Se desarrolló un equipo de caracterización de filtros multi e hiperespectrales que constó de una platina motorizada y de un microespectrómetro, como se explicó en el Capítulo 3. Para ello se realizó previamente una prueba de concepto, con equipamiento disponible en el laboratorio. En dicho prototipo preliminar se desarrolló el \textit{software} automatizado de adquisición de los espectros de transmisión de cada una de las bandas del filtro y una interfaz gráfica para poder visualizar los resultados de las mediciones.

Se desarrolló el \textit{software}, la electrónica y la mecánica a partir de las piezas 3D diseñadas de una platina motorizada con la longitud de recorrido suficiente como para poder medir el espectro de transmisión en cualquier punto del filtro deseado, superando de esta manera la limitación que se tenía con la plataforma motorizada de Thorlabs utilizada en el prototipo preliminar. El código y las piezas 3D fueron publicadas online para que cualquier laboratorio pueda replicar la platina y adaptarla a los requerimiento específicos. Al momento de escribir esta tesis, no se registraban platinas de estas características desarrolladas en el país por lo que este prototipo representa una base fundamental para el posterior desarrollo de platinas de microscopía de estas características.

Se montó y se desarrolló un método de alineación de un microespectrómetro, técnica de microscopía de la cual no se tenían antecedentes en el país. Se determinó experimentalmente la resolución espacial lateral del microespectrómetro cuyo mínimo valor obtenido fue igual a $(11 \pm 1)~\mu m$, valor que se solapa con el valor esperado teóricamente de 10 $\mu m$, de acuerdo a la estimación realizada por medio de la óptica geométrica, considerando la magnificación del microespectrómetro y el valor del diámetro del \textit{core} de la fibra del espectrómetro.

Se adquirieron los espectros de transmisión de cada banda del filtro, obteniéndose resultados que se solapan con los reportados por el fabricante, a excepción de la banda pancromática donde la diferencia entre el valor medido y el reportado fue del 2\%. Estos resultados motivaron la implementación del microespectrómetro para medir los espectros de transmisión de los filtros hiperespectrales que tienen cientos de bandas contiguas entre sí de un ancho en longitud de onda muy estrecho. Estas mediciones están previstas para ser realizadas ni bien se termine la cuarentena debido al COVID-19.

Se integró una cámara al equipo para poder tener una retroalimentación de las mediciones de los espectros de transmisión de los defectos con una imagen digital. Al mismo tiempo, se incorporó de esta manera al microespectrómetro las prestaciones de cualquier microscopio digital comercial y de esta manera se podría implementar el algoritmo de detección de los defectos en las imágenes adquiridas con el equipo desarrollado.

Se caracterizaron espectralmente las manchas y los huecos. Para las manchas se observó experimentalmente que el espectro de transmisión sufre una disminución de la intensidad total transmitida pero que no se modifica la forma funcional del espectro esperado de la banda sobre la que se midió el defecto. Por el contrario, para los huecos se observó que el espectro de transmisión de la fuente de luz contribuye al espectro medido en el centro del hueco lo que representa una distorsión del espectro de diseño del filtro. Se determinó experimentalmente que la presencia de un hueco modifica la huella espectral de diseño del fabricante de la banda asociada al filtro y además produce una variación de la intensidad de luz total transmitida. 

Por último, como continuación del trabajo aquí presentado está previsto armar un arreglo experimental para poder caracterizar el filtro en su posición final en las cámaras de vuelo del satélite y de esa forma poder caracterizar los efectos de los distintos tipos de defectos aquí caracterizados en el proceso de formación de imágenes.
Por último, como continuación del trabajo aquí presentado está previsto armar un arreglo experimental para poder caracterizar el filtro en su posición final en las cámaras de vuelo del satélite y de esa forma poder caracterizar los efectos de los distintos tipos de defectos aquí caracterizados en el proceso de formación de imágenes.