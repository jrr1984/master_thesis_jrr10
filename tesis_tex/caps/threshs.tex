\newcommand{\mychapter}[2]{
    \setcounter{chapter}{#1}
    \setcounter{section}{0}
    \chapter*{#2}	
}

\singlespacing
\mychapter{0}{Apéndice: \texttt{Thresholds}}
\begin{tikzpicture}[remember picture,overlay]
\node[anchor=east,inner sep=0pt] at (current page text area.east|-0,3cm) {\includegraphics[height=3cm]{Figs/scikitlogo.png}};
\end{tikzpicture}
\label{chap:apendthresh}
\spacing{1.5}

En este apéndice se brinda un resumen explicativo de los \textit{thresholds} que son utilizados en el método \textit{try\_all\_threshold} de la librería \textit{scikit-image} del lenguaje \textit{python} \cite{van2014scikit}.
Código fuente: \href{https://github.com/scikit-image/scikit-image/blob/master/skimage/filters/thresholding.py}{\faGithub}.

\begin{itemize}
\justifying
\item \texttt{Isodata}: Umbral basado en el histograma de la imagen, que utiliza el método de Ridler-Calvard o \textit{inter-means} \cite{4310039}. El umbral resultante del método satisface la siguiente igualdad:

\begin{equation*}
threshold = \frac{image[image <= threshold].mean() +
             image[image > threshold].mean()}{2.0}
\end{equation*}
Es decir, el umbral resultante constituye un valor de intensidad que separa la imagen en dos grupos de píxeles (\textit{background} y \textit{foreground}), donde la intensidad del umbral está a medio camino entre las intensidades medias de estos grupos. Para imágenes cuyos elementos de matriz son enteros, la igualdad anterior está acotada superiormente por el valor igual a 1 de intensidad y, para imágenes cuyos elementos de matriz son de punto flotante, la igualdad se mantiene dentro del ancho del \textit{bin} del histograma.



\item \texttt{Minimum}:  Umbral basado en el histograma de la imagen, que es calculado y suavizado con un filtro uniforme de forma iterativa hasta que se puedan determinar únicamente dos máximos locales en el mismo. El método puede no converger antes de que se hayan realizado el número completo de iteraciones indicado en el argumento de la función. Ahora bien, si el método converge, se toma como umbral al mínimo valor de intensidad que se encuentre en el rango de intensidades comprendido por los dos máximos \cite{GLASBEY}.

\item \texttt{Otsu}: Este algoritmo supone que la imagen que se va a binarizar contiene dos clases de píxeles únicamente asociados al \textit{background} y al \textit{foreground}. Se calcula el umbral óptimo que separa esas dos clases de píxeles a partir de la maximización de la varianza de los valores medios de las clases de píxeles (\textit{inter-class variance}) en simultáneo con la minimización de los valores medios de las varianzas de cada clase de píxeles (\textit{intra-class variance}) \cite{otsuu}.

\item \texttt{Li}: Este umbral es determinado a partir de un algoritmo iterativo que determina el óptimo \textit{threshold} a partir del cálculo de la pendiente de la entropía cruzada (\textit{cross entropy}) entre el \textit{foreground} y el valor medio del \textit{foreground} y, entre el \textit{background} y el valor medio del \textit{background} \cite{LIl}.

\item \texttt{Mean}: Este umbral es determinado a partir del valor medio de los valores de intensidad de los elementos de matriz de la imagen. Es utilizado por otros métodos como umbral semilla de un algoritmo iterativo \cite{GLASBEY}. 

\item \texttt{Triangle}: El algoritmo para determinar el \textit{threshold Triangle} es un método geométrico que no puede determinar si los datos están sesgados hacia un lado u otro (\textit{background} ó \textit{foreground}), pero supone un pico máximo cerca de un extremo del histograma y busca hacia el otro extremo. Esto causa un problema en ausencia de información previa del tipo de imagen a procesar, o cuando el máximo no está cerca de uno de los extremos del histograma, ya que se puede decaer en dos posibles regiones del umbral entre ese máximo y los extremos. El algoritmo finalmente determina en qué lado del pico máximo los datos van más lejos y se determina el umbral dentro del rango de mayor extensión \cite{triang}.

\item \texttt{Yen}: El algoritmo del criterio de máxima entropía (\textit{maximum entropy criterion} (MEP) determina un umbral tal que la información total provista por el \textit{foreground} y por el \textit{background} es máxima y esto se puede observar en que la distribución de los valores de intensidad de los elementos de matriiz de la imagen se vuelve lo más uniforme posible.


To reduce the computational complexity of the MaxEntropy algorithm,
the maximum correlation criterion (MCC) was proposed by Yen
in 1995 (Yen et al., 1995) to determine the optimal threshold T∗, the
construction of which, i.e., using the correlation instead of the entropy,
comes from the theory of chaos and fractal theory. Similar to the MEC,
the basic principle of the MCC is to choose a threshold so that the total

implements thresholding based on a maximum correlation criterion [12] as a more computationally efficient alternative to entropy measures.

[\cite{yen}, \cite{yen2}].

\end{itemize}


