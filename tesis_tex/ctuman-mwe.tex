\documentclass{ctuthesis}

\ctusetup{
	xdoctype = B,
	xfaculty = FCEyN,
	mainlanguage = czech,
	titlelanguage = czech,
	title-english = {Planting Uranium},
	title-czech = {Construcción e implementación de un sistema integral de 
	caracterización de filtros ópticos de interferencia de banda utilizados en 
	cámaras 
	hiperespectrales.},
	department-czech = {Departamento de Física},
	author = {Juan Reto Reynal},
	supervisor = {Prof. Dr. Hernán Grecco},
	supervisor-address = {Laboratorio de Electrónica Cuántica, DF, FCEyN, UBA.},
	month = 4,
	year = 2019,
}

\ctuprocess

\begin{abstract-english}
We develop \ldots
\end{abstract-english}

\begin{abstract-czech}
Rozvíjíme \ldots
\end{abstract-czech}

\begin{document}

\maketitle
\renewcommand{\chaptername}{Capítulo}
\chapter*{Plan de Tesis - Juan Reto Reynal}


\textsc{Título:} Construcción e implementación de un sistema integral de 
caracterización de filtros ópticos de interferencia de banda utilizados en 
cámaras 
hiperespectrales.


\hspace{-0.4cm}\textsc{Alumno:} Juan Reto Reynal L.U. 777/12.

\hspace{-0.4cm}\textsc{Director:} Dr. Hernán E. Grecco, Inv. Adj. CONICET, Prof. Adj. UBA.

\hspace{-0.4cm}\textsc{Lugar de trabajo:} LEC, Departamento de Física, FCEyN, UBA.


\section*{Objetivo general}
%conectar càmaras hiperespectrales con filtros y sus aplicaciones
%Como objetivo general se propone desarrollar métodos para optimizar la
%adquisición de imágenes hiperespectrales en las cuales se recopila y procesa, 
%con
%resolución espacial, información en un rango del espectro electromagnético. 
%Utilizando
%un modelo realista de adquisición que incluya parámetros tales como la 
%respuesta
%espectral del detector y la óptica así como también las características del 
%objeto
%observado, se optimizarán los métodos de sensado y deconvolución espectral. En
%particular se estudiará el efecto de variables físicas de relevancia, como por 
%ejemplo
%número de fotones, resolución espectral, resolución espacial y relación señal 
%ruido. En
%este trabajo, se estudiarán en particular la toma de imágenes hiperespectrales
%mediante sensores remotos y sus aplicaciones en la industria satelital.

%semrock:
Los filtros ópticos utilizados en la construcción de cámaras hiperespectrales y 
multiespectrales resultan fundamentales en aplicaciones como la microscopía de 
fluorescencia \cite{Grecco2016} y en la espectroscopía de reflectancia 
utilizada para la toma de 
imágenes multiespectrales de la superficie de la Tierra \cite{Hogg2008}. En 
estas aplicaciones 
se distinguen claramente dos señales: una señal de entrada que viene dada por 
la iluminación (excitación) de una muestra ó la incidencia de la luz sobre la 
Tierra y, una señal de respuesta dada por la emisión en una muestra 
biológica ó la reflexión en el caso de la superficie de la Tierra.

Ambas señales no son solo espectralmente distintas sino que además difieren en 
su intensidad: la señal de entrada puede ser un millón de veces (o más aún) más 
intensa que la señal de respuesta (reflexión/emisión). En consecuencia, la 
capacidad de los filtros ópticos para transmitir ciertas longitudes de onda 
deseadas y bloquear el resto, resulta crítica para estas aplicaciones. Ahora 
bien, cuando el ancho de banda centrado en una cierta longitud de onda central 
que se desea transmitir es muy estrecho, las mediciones de las características 
espectrales y de transmisión de dichos filtros no suelen ser determinadas con 
precisión. Más aún, si los filtros son construidos especialmente por un 
proveedor (\textit{custom-made}) para una cierta aplicación específica, resulta 
fundamental su caracterización para decidir si incorporar o descartar dichos 
filtros a la carga útil de la aplicación en cuestión.

En el presente proyecto se propone desarrollar un sistema integral de 
caracterización de filtros de interferencia de banda con la capacidad de 
decidir si un filtro está apto o no para ser integrado a las cámaras 
hiperespectrales incorporadas en los satélites desarrollados por la empresa 
Satellogic.


%Optical filters play an important role in enabling applications such as 
%%%fluorescence
%microscopy and Raman spectroscopy. In these applications there are two 
%distinct types of
%beams: the illumination (or excitation) beam and the signal (or emission) 
%beam. 

%Not only are
%these beams spectrally distinct, but also they differ significantly in their 
%intensity – the signal
%beam can be a million times (or more) weaker than the illumination beam. 
%Therefore, the ability
%of filters to selectively transmit desired wavelengths of light while blocking 
%unwanted light is
%critical. The performance of such filters is determined by their spectral 
%characteristics, including
%transmission efficiency of the signal and attenuation (or blocking) of the 
%illumination light and
%undesirable emission wavelengths. In particular, often it is critical for 
%filters to transition from
%deep blocking to high transmission over a very short wavelength range, leading 
%to steep and
%deep spectral edges.
%However, due to limitations of standard metrology techniques, the
%measured spectral characteristics of thin-film interference filters are 
%frequently not determined
%accurately, especially when there are steep and deep edges.
%In this article we explore limitations to accurate filter spectrum 
%measurements with
%standard metrology techniques, and show how these limitations can be managed 
%by a better
%understanding of the limitations as well as more sophisticated measurements 
%when necessary.

\section*{Objetivos específicos del proyecto}
\begin{itemize}
	
	\item \texttt{Objetivo 0 - Abril:} Revisión del estado del arte.
	\item \texttt{Objetivo 1 - Mayo:} Armado de distintos setups de iluminación 
	o detección para distintas longitudes de onda para analizar los espectros 
	de transmisión de los filtros.
	\item \texttt{Objetivo 2 - Junio:} Armado de distintas configuraciones 
	experimentales controlando distintas formas de iluminar los filtros
	para poder encontrar los defectos.
	\item \texttt{Objetivo 3 - Agosto:} Construcción e implementación de un primer prototipo de un sistema integral que pueda decidir si un filtro está apto o no para ser integrado al satélite.
	\item \texttt{Objetivo 4 - Septiembre:} Establecer control de una de las cámaras de la empresa Satellogic para poder adquirir las
	imágenes. Procesar las imágenes tomadas haciendo HDR y búsqueda de características.
	\item \texttt{Objetivo 5 - Octubre:} Armado de un posible setup 
	experimental para poder caracterizar el filtro en su posición final
	en las cámaras de vuelo del satélite.
\end{itemize}
\section*{Antecedentes}
La captura de imágenes hiperespectrales consiste en colectar y procesar 
información en longitudes de onda específicas del campo electromagnético. El 
objetivo es obtener el espectro para cada pixel en la imagen de una escena, con 
el propósito de hallar objetos, identificar materiales y sustancias, o detectar 
procesos.
Los sensores hiperespectrales colectan información como un  set de imágenes. Cada una de ellas representa un rango estrecho de longitudes de onda del espectro electromagnético, el cual se conoce como banda espectral. Estas imágenes se combinan para formar un cubo de datos tridimensional (x,y,$\lambda$) para el procesamiento y análisis, donde x e y representan dos dimensiones espaciales de la escena, y $\lambda$ representa la dimensión espectral (la cual comprende a un cierto rango de longitudes de onda).
En los distintos sensores hiperespectrales, las longitudes de onda pueden ser separadas por diferentes tipos de filtros, o mediante el uso de instrumentos que sean sensibles a determinadas longitudes de onda, como por ejemplo interferómetros. La forma en la que las distintas longitudes de onda se combinan en un mismo pixel está intrínsecamente relacionada con el diseño particular del sensor que se emplee. Por este motivo, existe una estrecha relación entre el modelo de sensor que se emplee y el algoritmo de deconvolución requerido. Existen diversos diseños de sensores de escaneo multiespectral, algunos de los cuales se encuentran detallados a continuación [1]. Un resumen de estos diseños y sus respectivas SNR se puede encontrar en la referencia [3]. Otros métodos basados en técnicas de escaneo se pueden hallar en las referencias [4], [5] y  [6]. 

Espectrómetro de escaneo puntual (Point Scanning Spectrometer): El espectro incidente es dispersado a lo largo de un arreglo lineal de elementos de detección, permitiendo tasas de lectura muy rápidas. La escena es escaneada  a través del sensor mediante el empleo de dos espejos galvanométricos (o solamente uno en caso de que el sensor mismo se esté desplazando).

Espectrómetro de barrido (Pushbroom Spectrometer): el input del sensor es una abertura lineal, cuya imagen se dispersa a través de una matriz bidimensional de detectores, de forma que todos los puntos a lo largo de la línea son muestreados simultáneamente. Para completar la dimensión espacial ortogonal a la línea, la escena es escaneada a través de la apertura de entrada. Esto puede tener la forma de objetos moviéndose a través de una cinta transportadora, el suelo desplazándose bajo un satélite o una plataforma espacial, o la escena siendo escaneada a través de la rendija de entrada mediante un espejo galvanométrico.

Cámara de filtros tuneables (Tunable Filter Camera): una cámara de filtros tuneables se caracteriza por utilizar un sistema de filtros ajustables, ya sea eléctrica o mecánicamente. Entre estos sistemas se pueden destacar las ruedas de filtros, los dispositivos Fabry - Perot mecánicamente ajustables [7] [8], y los filtros ajustables tanto de cristal líquido (LCTF) [9] como los acusto ópticos (AOTF) [10], entre otros. Los tiempos de respuesta en el ajuste de los distintos filtros van desde $\approx$1 s para la rueda de filtros, $\approx$50 a 500 ms para el LCTF y el Fabry Perot mecánicamente ajustable, y $\approx$10 a 50 $\mu$s para el AOTF.

Espectrómetro de proyección de transformada de Fourier (Imaging Fourier Transform Spectrometer): se caracteriza por escanear un espejo de un interferómetro de Michelson con el objetivo de obtener mediciones para múltiples diferencias de camino óptico [11] [12]. Una alternativa más reciente es el espectrómetro de proyección de transformada de Fourier birrefringente, desarrollado por Harvey y Fletcher - Holmes, que cuenta con la ventaja de ser menos sensible a las vibraciones. [13]

Espectrómetro hiperespectral por tomografía computada (Computed Tomography Hyperspectral Imaging Spectrometer): es un dispositivo de escaneo similar a la técnica de snapshot CTIS, pero cuenta con la ventaja de ser capaz de colectar proyecciones provenientes de una mayor cantidad de ángulos, de forma que la data reconstruida tenga menos artefactos. La desventaja es que el detector no es usado eficientemente en comparación con otros métodos. [14]

Espectrómetro lineal de apertura codificada (Coded Aperture Line-Imaging Spectrometer): a pesar de que la espectrometría de apertura codificada comenzó como un método para escanear una apertura codificada a través de la rendija de entrada de un espectrómetro convencional, ha sido adaptada a arreglos bidimensionales modernos de detectores, dando lugar a una mejora en la relación señal - ruido. [15] [16]

El tamaño espacial de los pixels en sensores multiespectrales e hiperespectrales es en general lo suficientemente amplio como para que distintas sustancias puedan contribuir al espectro medido por un solo pixel. El objetivo de los algoritmos de deconvolución espectral consiste en extraer de un espectro los materiales constituyentes en la mezcla, así como las proporciones en las que aparecen. 
La deconvolución espectral es el procedimiento mediante el cual el espectro medido por un pixel es descompuesto en una colección de espectros constituyentes, endmembers, y un set de las correspondientes fracciones, abundancias, que indican la proporción de cada endmember presente en el pixel. Esto es importante en numerosos escenarios en los que el detalle a nivel subpixel es apreciable, los cuales pueden ir desde el campo de la microscopía de fluorescencia [17] (donde se detallan aplicaciones como Timelapse imaging [18] [19] y FRET [20], entre otras) hasta la toma de imágenes satelitales mediante sensores remotos. En el primer caso, los endmembers y las abundancias se pueden asociar a los fluoróforos y canales que se empleen, mientras que en las aplicaciones satelitales los endmembers normalmente corresponden a objetos macroscópicos en la escena, tales como agua, tierra, metal, o cualquier material natural o hecho por el hombre.
El proceso de deconvolución de principio a fin es en realidad una concatenación de tres procedimientos distintos, cada uno con objetivos específicos. La reducción dimensional reduce la cantidad de datos con el objetivo de disminuir la carga computacional en los pasos de procesamiento subsecuentes. La determinación de endmembers estima el set de distintos espectros que componen los píxeles mixtos en la escena. La etapa de inversión consta de proveer estimaciones de las abundancias fraccionales para los endmembers en cada pixel. 
La examinación del proceso de deconvolución se puede estudiar mediante tres criterios que categorizan estos algoritmos. 1. La interpretación de la data indica cómo un algoritmo interpreta el espectro combinado de un pixel. Principalmente se pueden distinguir dos clases de algoritmos, los estadísticos y los no estadísticos. 2. La descripción de la aleatoriedad indica cómo un algoritmo incorpora la aleatoriedad de los datos. Se pueden diferenciar según este criterio los métodos paramétricos de los no paramétricos. 3. Por último, el criterio de optimización indica cuál es la función objetiva que se está optimizando. Según este criterio, se pueden diferenciar los algoritmos que optimizan funciones tales como Squared error, Non squared error, Maximum a posteriori, Maximum likelihood, entre otras. [2]
\section*{Metodología}
Dentro del marco de esta investigación se realizarán las siguientes actividades. A través del estudio de papers y patentes acerca de los sistemas de sensado multiespectral existentes, se trabajará con los distintos algoritmos existentes de deconvolución espectral con el objetivo de familiarizarse con el estado del arte y conocer sus aplicaciones y restricciones. Se estudiará la aplicabilidad de cada uno de estos métodos en distintos casos. Para esto se cuenta con imágenes satelitales multiespectrales provistas por Satellogic. 
Paralelamente, se desarrollará un método de generación de imágenes satelitales multiespectrales artificiales que simule las características de la imagen relevantes para su adquisición. Las imágenes generadas se utilizarán para probar los algoritmos de deconvolución en cuestión. Finalmente se emplearán estos métodos en imágenes reales multiespectrales obtenidas de un microscopio confocal. 
El siguiente paso será el desarrollo de un algoritmo de deconvolución espectral óptimo que tenga en cuenta las variables físicas de relevancia en el sensado de imágenes, tales como el número de fotones, resolución espectral, resolución espacial y relación señal - ruido. En primer caso este estudio se hará para un caso genérico (lo cual tiene aplicaciones de suma importancia en microscopía, por ejemplo), y posteriormente se analizará el caso particular de sensores remotos, el cual presenta aplicaciones en la toma de imágenes satelitales. Las variables de interés serán determinadas en este caso mediante el análisis de imágenes satelitales existentes, las cuales serán provistas por Satellogic.
El software a desarrollar se encuentra intrínsecamente relacionado con el método de sensado que se emplee en la toma de las imágenes. El análisis del algoritmo desarrollado permitirá estudiar cuál es el método óptimo para la toma de imágenes multiespectrales, pudiendo aplicarse los resultados tanto a la microscopía como a la toma de imágenes satélitales por sensores remotos. El algoritmo de deconvolución se desarrollará en el lenguaje Python. 

\section*{Actividades asociadas al objetivo 1 - Mayo: Armado de distintos 
setups de iluminación 
	o detección para distintas longitudes de onda para analizar los espectros 
	de transmisión de los filtros.}

It is well known that in order to effectively suppress undesired light, the 
blocking
specification of an optical filter must be typically many orders of magnitude 
higher than that of
transmission. Optical Density – or OD, as it is commonly called – is a 
convenient tool to
describe the transmission of light through a highly blocking optical filter, or 
one with extremely
low transmission.
OD is simply defined as the negative of the logarithm (base 10) of the
transmission, where the transmission varies between 0 and 1; that is OD = – log 
10 (T).
The actual blocking provided by a filter is determined not only by its designed 
spectrum, but
also by physical imperfections of the filter, such as pinholes generated during 
the thin-film
coating process, dirt and other surface defects, or flaws in the filter 
mounting. Pinholes can
allow light to pass through the filter unblocked – a single 10 $\mu$m pinhole 
(that 
penetrates
completely through the coating) limits the blocking of a 10 mm diameter beam to 
at most OD 6,
regardless of the designed level of blocking of the filter spectrum. Other 
surface and mounting
imperfections can cause scattered light that ''leaks'' through the filter due 
to 
a shift of the
spectrum to a region of high transmission for scattered light at high angles of 
incidence, or via unblocked paths near the edges or mounting. Therefore, it is 
important to evaluate the blocking
performance of filters after they have been fully manufactured into finished 
products.
Generally commercially available spectrophotometers are used to measure the 
transmission
and OD spectral performance of optical filters. However, these instruments can 
have significant
limitations when the optical filters have high edge steepness and/or very deep 
blocking.
The principle of operation of a typical spectrophotometer is illustrated in 
Figure 1.
A
monochromator contains a diffraction grating to disperse light from a broadband 
source (usually
a quartz-tungsten-halogen (QTH) or arc lamp) into a range of angles, and then 
uses an
adjustable slit to select a narrow band of wavelengths. This 
quasi-monochromatic probe beam
is then directed toward the sample (test) filter. In a dual-beam 
spectrophotometer (as shown
here), either a beamsplitter is used to split the probe beam into reference and 
measurement
beams, or the light is alternated in time between the reference and measurement 
paths at a
relatively high rate.
The reference beam goes directly to the detector, though it might be
attenuated with a calibrated neutral density filter depending on the blocking 
level of the sample
filter. The measurement beam passes through the sample filter and then impinges 
on the
detector. The filter spectrum, or its transmission (or blocking) specification 
as a function of
wavelength, is established from the ratio of the signals from the two beams as 
the wavelength is
scanned over a broad range. The spectrum can also be obtained from a single-beam
spectrophotometer, except in this case the reference signal is generated 
without the sample
filter in the light path and then the measurement signal is generated by 
inserting the sample
filter into the light path (as in Fig. 4 below).
\section*{Actividades asociadas al objetivo 2 - Junio:}
\section*{Actividades asociadas al objetivo 3 - Agosto:}
\section*{Actividades asociadas al objetivo 4 - Septiembre:}
\section*{Actividades asociadas al objetivo 5 - Octubre:}
\section*{Factibilidad}
El plan de trabajo propuesto para esta tesis se enmarca dentro del área de óptica y fotofísica. La tesista será guiada por un investigador con la experiencia en la temática necesaria para llevar el proyecto adelante, siendo el director del Laboratorio de Electrónica Cuántica, en el cual continuamente se aplican técnicas de deconvolución espectral en el área de la microscopía.
La tetesista se encuentra cursando la última materia de la Licenciatura en Ciencias Físicas. Durante Laboratorio 6 y 7, trabajó en el armado y optimización de un microscopio confocal de fluorescencia. Adicionalmente, cursó la materia optativa “Microscopías de fluorescencia”, por lo que posee conocimientos en microscopía, óptica y fotofísica que serán necesarios para determinar las variables físicas de interés en la optimización de la toma de imágenes hiperespectrales. Posee también conocimientos de programación que serán necesarios para realizar la implementación de los algoritmos de deconvolución.
Adicionalmente, se cuenta con imágenes satelitales provistas por Satellogic, lo que permitirá testear el funcionamiento de los algoritmos con imágenes de distinta calidad, en aplicaciones satelitales. 
\section*{Referencias}
1. [Hagen 2013] Nathan Hagen , Michael W. Kudenov. Review of snapshot spectral imaging technologies. Optical Engineering 52(9), 090901.

2. [Keshava 2003] Nirmal Keshava. A Survey of Spectral Unmixing Algorithms. Lincoln Laboratory Journal, Vol. 14, 1, 2003.

3. [Sellar 2005] R. G. Sellar and G. D. Boreman, “Comparison of relative signal-tonoise
ratios of different classes of imaging spectrometer,” Appl. Opt. 44(9), 1614–1624 (2005).

4. [Eismann 2012] M. T. Eismann, Hyperspectral Remote Sensing, SPIE Press, Bellingham, WA (2012).

5. [Harvey 2000] A. R. Harvey et al., “Technology options for imaging spectrometry,”
Proc. SPIE 4132, 13–24 (2000).

6. [Prieto 2008] X. Prieto-Blanco et al., “Optical configurations for imaging spectrometers,” Comput. Intell. Rem. Sens. 133, 1–25 (2008).

7. [Atherton 1981] P. D. Atherton et al., “Tunable Fabry-Perot filters,” Opt. Eng. 20(6),
806–814 (1981).

8. [Antila 2012] J. Antila et al., “Spectral imaging device based on a tuneable MEMS
Fabry-Perot interferometer,” Proc. SPIE 8374, 83740F (2012).

9. [Gupta 2008] N. Gupta, “Hyperspectral imager development at Army Research
Laboratory,” Proc. SPIE 6940, 69401P (2008).

10. [Poger 2001] S. Poger and E. Angelopoulou, “Multispectral sensors in computer
vision,” Technical Report CS 2001-3, Stevens Institute of Technology (2001).

11. [Potter 1972] A. E. Potter, “Multispectral imaging system,” U.S. Patent No. 3702735 (1972).

12. [Descour 1996] M. R. Descour, “The throughput advantage in imaging Fouriertransform spectrometers,” Proc. SPIE 2819, 285–290 (1996).

13. [Harvey 2004] A. R. Harvey and D. W. Fletcher-Holmes, “Birefringent Fourier-transform imaging spectrometer,” Opt. Express 12(22), 5368–5374 (2004).

14. [Mooney 1995] J. M. Mooney, “Angularly multiplexed spectral imager,” Proc. SPIE
2480, 65–77 (1995).

15. [Fernandez 2007] C. Fernandez et al., “Longwave infrared (LWIR) coded aperture
dispersive spectrometer,” Opt. Express 15(9), 5742–5753 (2007).

16. [Gehm 2008] M. E. Gehm et al., “High-throughput, multiplexed pushbroom hyperspectral microscopy,” Opt. Express 16(15), 11032–11043 (2008).

17. [Zimmermann 2005] Timo Zimmermann. Spectral Imaging and Linear Unmixing in Light Microscopy. Adv Biochem Engin/Biotechnol (2005) 95: 245– 265

18. Shima DT, Scales SJ, Kreis TE, Pepperkok R (1999) Curr Biol 9:821 

19. Ellenberg J, Lippincott-Schwartz J, Presley JF (1999) Trends Cell Biol 9:52

20. [Wouters 2001] Wouters FS, Verveer PJ, Bastiaens PI (2001) Trends Cell Biol 11:203
\chapter{Conclusion}

Lorep ipsum \cite{doe}

\begin{thebibliography}{1}

\bibitem{Grecco2016} Hernán E. Grecco, Sarah Imtiaz y Eli Zamir. Multiplexed
imaging of intracellular protein networks. 2016.

\bibitem{Hogg2008} David W. Hogg y Dustin Lang. “Astronomical imaging: The
	theory of everything”. AIP Conference Proceedings. Vol. 1082.
	2008, págs. 331-338. arXiv: 0810.3851.

\end{thebibliography}

\end{document}

